\section{Introduction}
\label{sec:intro}

%\concept{building secure systems is hard}
Building software systems is a challenging, error-prone task.
%
As the number of recent vulnerabilities in SSL libraries and the
severity of bugs---where a single line change can introduce a
vulnerability~\cite{goto}---show, building secure systems is harder
still.
%
Unfortunately, as we have learned over again (e.g., recently by the
popular bug in GitHub's form-submission code~\cite{gitfail}) such
security-related bugs are inevitable even if developers are careful
and use safe, high-level languages.
%
Indeed, this is because only a small fraction of programmers have the
right mindset and experience to write secure code.
%


%\concept{ifc to the rescue}
How then can we expect the average developer to build secure systems?
%
One approach to bridging this security gap is to use decentralized
information-flow control (DIFC)~\cite{myers:dlm,
sabelfeld:language-based-iflow}.
%
DIFC tracks and controls the flow of information through a system, in
order to preserve data confidentiality and integrity.
%
By ensuring that the policy associated with a piece data is always
enforced, most code in a DIFC system can be considered untrustworthy.
%
Indeed, typical DIFC applications are composed of a small trustworthy
component, where the security policy is specified, and the rest of the
application logic, which is considered untrustworthy.
%

%
%\concept{example}
Consider, for instance, a conference review system where reviewers are
expected to be anonymous and users in conflict with a paper are
prohibited from reading specific committee comments.
%
Here, the site administrator is trusted to specify such policies on
reviews. 
%
However, the rest of the application, which may for instance fetch
data from the network, read reviews, etc., can be built by
untrustworthy developers (even a conflicting reviewer)---the
underlying runtime system ensures that the confidentiality and
integrity of a user's reviews will be preserved.
%
In a DIFC system, bugs that appear in this crux of the application
simply \emph{that}---bugs---not vulnerabilities.

In this demonstration, we describe one such DIFC system, called
LIO~\cite{lio, concurrent-lio}.
%
LIO is implemented as a Haskell library, and leverages Haskell's
monadic approach to encoding side-effects as a way to control how (and
if) information enters/exists the system.
%
Specifically, LIO provides an \hs|LIO| monad, in which
all side-effects are mediated according to DIFC.
%
Importantly, however, a computation can perform arbitrary, complex
effects, as long as it does not violate the confidentiality or
integrity of data.
%
(Indeed, LIO automatically disallows effects that would violate
confidentiality or integrity.)

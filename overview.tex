\section{Overview}
\label{sec:overview}

DIFC systems such as LIO track and control the propagation of
information by associating a \emph{label} with every piece of data.
%
(While LIO is polymorphic in the label model, we focus on LIO with
DCLabels~\cite{dclabels}, henceforth just labels.)
%
A label encodes a security policy as a pair of positive boolean
formulas over \emph{principals} specifying who may read or write data.
%
For example, a review labeled \hs|"alice" \/ "bob" %% "bob"| specifies
that the review can be read by user \hs|"alice"| or \hs|"bob"|, but
may only be modified by \hs|"bob"|.
%
Indeed, such a label may be associated with \hs|"bob"|'s review, for a
paper that both \hs|"bob"| and \hs|"alice"| are reviewing.
%

Labels are partially ordered according to a {\em can flow
  to} relation $\canflowto$: for any labels $L_A$ and $L_B$, if $L_A
  \canflowto L_B$ then the policy encoded by $L_A$ is \emph{upheld}
  by that of $L_B$.
%
For example, data labeled $L_A =$ \hs|"alice" \/ "bob" %% "bob"| can
be written to a file labeled $L_B =$ \hs|"bob" %% "bob"| since $L_B$
preserves the secrecy of $L_A$.
%
In fact, $L_B$ is \emph{more} restrictive, as only
\hs|"bob"|---not both \hs|"alice"| and \hs|"bob"|---can read the file,
and, indeed, until \hs|"alice"| submits her review we may wish to
associate this label with \hs|"bob"|'s review as to ensure that she
cannot read it.
%
Conversely, $L_B \not\canflowto L_A$, and thus data labeled $L_B$
cannot be written to an object labeled $L_A$ (data secret to
\hs|"bob"| cannot be leaked to an object that \hs|"alice"| can also
read).


\subsection{Specifying policies with DCLabels}

\begin{minted}[frame=lines]{haskell}
policy rev = do
  let author = reviewAuthor rev
  reviewers <- findReviewers $ reviewId rev
  makePolicy $ do
    readers ==> author \/ reviewers
    writers ==> author
\end{minted}

\subsection{Restricting code to a safe sublanguage}
\subsection{Fine grained labeling}
\subsection{Automatic data labeling for Web applications}

\section{Introduction}
\label{sec:intro}

Haskell provides many language features that can be used to reduce the
damage caused by any particular piece of code.
%
Notable among these are the strong static type system and module
system.
%
The type system, in addition to reducing undefined behavior, can be
used to distinguish between pure and side-effecting computations,
i.e., computations that respectively can and cannot affect the
``external world,'' while the module system can be used to enforce
abstraction (e.g., by restricting access to constructors).\footnote{
  Here, we refer to the safe subset of the Haskell language---without
  \texttt{unsafePerformIO}, etc.---as enforced by
  the \emph{Safe Haskell} extension~\cite{safehaskell}.
}
%
Unfortunately, even in such a high-level, type-safe language, building
software systems is an error-prone task and only a few programmers are
equipped to write secure code,
 
Consider, for instance, a conference review system where reviewers are
expected to be anonymous and users in conflict with a paper are
prohibited from reading specific committee comments.
%
When building such a system, if we import a library function that
performs IO, we risk violating these guarantees---if the code is
malicious, it may, for instance, read reviews from the database and
leak them to a public server.
%
How, then, can we restrict the effects of a computation, without
imposing that it not perform \emph{any} side-effects?\footnote{
  Indeed, the pure code limitation is unreasonable once we wish to
  further treat our own code as untrustworthy, as to prevent leaks due
  to bugs.
}

One approach is to restrict computations to a particular monad---one
other than \texttt{IO}---for which we can control effects.
%
In this demonstration, we describe the LIO library which implements
one such monad, called \texttt{LIO} (\emph{Labeled IO}).
%
Effects in the \texttt{LIO} are mediated according to decentralized
information flow control (DIFC) policies~\cite{myers:dlm,
sabelfeld:language-based-iflow}.

\section{Introduction}
\label{sec:intro}

Haskell provides many language features that can be used to reduce the
damage caused by any particular piece of code.
%
Notable among these are the strong static type system and module
system.
%
The type system, in addition to reducing undefined behavior, can be
used to distinguish between pure and side-effecting computations,
i.e., computations that respectively can and cannot affect the
``external world,'' while the module system can be used to enforce
abstraction (e.g., by restricting access to constructors).
%

However, real world code sometimes requires that we break these
abstraction barriers (e.g., to provide a fast string library) and
Haskell provides certain features (e.g., \verb|unsafePerfromIO|) that
facilitates this.
%
Unfortunately this means that if we integrate any untrusted code,
our program cannot provide any security guarantees---if the untrusted
code is malicious it can leverage any of these unsafe features to
cause havoc.
%
Hence, to distinguish between the full (unsafe) Haskell language and
the safe subset, Terei et al. introduced \emph{Safe
Haskell}~\cite{safehaskell}.

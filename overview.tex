\section{Overview}
\label{sec:overview}

DIFC systems such as LIO track and control the propagation of
information by associating a \emph{label} with every piece of data.
%
(While LIO is polymorphic in the label model, we focus on LIO with
DCLabels~\cite{dclabels}, henceforth just labels.)
%
A label encodes a security policy as a pair of positive boolean
formulas over \emph{principals} specifying who may read or write data.
%
For example, a review labeled \hs|"alice" \/ "bob" %% "bob"| specifies
that the review can be read by user \hs|"alice"| or \hs|"bob"|, but
may only be modified by \hs|"bob"|.
%
Indeed, such a label may be associated with \hs|"bob"|'s review, for a
paper that both \hs|"bob"| and \hs|"alice"| are reviewing.
%

Our LIO library associates labels with various Haskell constructs.
%
For example, we provide labeled alternatives of \hs|IORef|, \hs|MVar|,
and \hs|Chan|, called \hs|LIORef|, \hs|LMVar|, and \hs|LChan|,
respectively.
%
Moreover, we provide an implementation of a filesystem that associates
persistent labels with files and a type, \hs|Labeled|, that is used to
associate a label with individual Haskell terms.
%
The latter, for example, is used to associate labels with reviews (e.g.,
as given by the type \hs|Labeled DCLabel Review|).

Labels on objects are partially ordered according to a {\em can flow
  to} relation $\canflowto$: for any labels $L_A$ and $L_B$, if $L_A
  \canflowto L_B$ then the policy encoded by $L_A$ is \emph{upheld}
  by that of $L_B$.
%
For example, data labeled $L_A =$ \hs|"alice" \/ "bob" %% "bob"| can
be written to a file labeled $L_B =$ \hs|"bob" %% "bob"| since $L_B$
preserves the secrecy of $L_A$.
%
In fact, $L_B$ is \emph{more} restrictive, as only
\hs|"bob"|---not both \hs|"alice"| and \hs|"bob"|---can read the file,
and, indeed, until \hs|"alice"| submits her review we may wish to
associate this label with \hs|"bob"|'s review as to ensure that she
cannot read it.
%
Conversely, $L_B \not\canflowto L_A$, and thus data labeled $L_B$
cannot be written to an object labeled $L_A$ (data secret to
\hs|"bob"| cannot be leaked to a file that \hs|"alice"| can also
read).

It is precisely this relation that is used by LIO when restricting the
effects performed by a computation in the \hs|LIO| monad.
%
In fact, the \hs|LIO| monad solely encapsulates the underlying
\hs|IO| computation and a label, called the \emph{current label}, that
tracks the sensitivity of the data that the computation has observed.
%
To illustrate the role of the current label, consider the code below
that reads \hs|"bob"|'s private review and tries to leak it into a
labeled references that \hs|"alice"| can read.
\begin{minted}[frame=lines, mathescape]{haskell}
-- Current label: public == True %% True
bobReview <- readFile "/reviews/bob/5.txt"
-- Current label: "bob" %% True
-- labelOf aliceRef == "alice" %% "alice"
writeLIORef aliceRef show bobReview
-- Fail: "bob" %% True $\not\canflowto$ "alice" %% "alice"
\end{minted}
%
Here, the current label is first raised by \hs|readFile| to reflect
the fact that information sensitive to \hs|"bob"| was incorporated
into the context.
%
Importantly, however, this label is also used to subsequently restrict
the effects performed by the computation; in this case, the
\hs|writeLIORef| action raises an exception to reflect that
the computation tried to write to a reference whose label
does not protect the review content.

In general, DIFC enforcement in LIO follows this approach of exposing
functions (e.g., \hs|writeLIORef|), which inspect the current label
and the label of object they are about to read/write as to uphold the
\emph{can flow to} relation; our definition for bind and return are trivial.
%
We solely rely on Haskell's monad support as a way to define a
sublanguage that enforces DIFC.
%
By ensuring (using Safe Haskell) that untrusted code is written in
this sublanguage, i.e., it cannot lift arbitrary \hs|IO| actions into
\hs|LIO|, we can incorporate arbitrary (untrusted) code to compute on
sensitive data.
%
For example, our conference review system can incorporate code
provided by users of the system without fear of leaking reviews or
reviewer identities, all while allowing the code to interact with the
external world.

A further important consequence of this approach to DIFC is that once
we have a sound core language, which, in the case of LIO, is both
concurrent and supports exceptions,\footnote{
  The presence of exceptions in the core calculus is very important,
  since it allows code to recover from DIFC violation
  attempts~\cite{Breeze, stefan:2012:arxiv-flexible}.
  %
  For example, the failure of the above code to write to a reference
  is not fatal---the untrusted code can recover and continue
  executing.
  
}
we can introduce many features by simply wrapping existing \hs|IO|
code.
%
As mentioned, LIO supports labeled alternatives to mutable references,
mutable variables, channels, files, databases, HTTP clients, etc.
%
Some of these features (e.g., the database) are crucial for
implementation applications such as the conference review system.

\section{Automatic data labeling for Web applications}

LIO guarantees that code executing in the \hs|LIO| monad cannot
violate the confidentiality and integrity restrictions imposed by
labels.
%
Unfortunately, assigning appropriate labels to data is challenging and
setting overly-permissive labels can amount to unexpected ``leaks.''
%
While using a simple label model such as DCLabels may help avoid
certain pitfalls, an alternative approach is clearly desirable.

In the context of web applications, we present an advancement towards
making DIFC policy-specification a mortal task.\footnote{
  We have considered the alternative approach of cloning MIT Prof.
  Nickolai Zeldovich.
}
%
Specifically, we demonstrate the declarative policy language, previously
developed for the Hails web framework~\cite{hails}.
%
In web applications, it is common for developers to specify the
application data model in a declarative fashion.
%
Hails leverages this design parading and the observation that, in many
web applications, the authoritative source for who should access data
resides in the data itself to provide developers with a means for
specifying the policy alongside the data model.
%

Consider the definition of the \hs|Review| data type used in our
conference review system:
\begin{minted}[frame=lines]{haskell}
data Review = Review { reviewId    :: ReviewId
                     , reviewPaper :: PaperId
                     , reviewOwner :: UserName
                     , reviewBody  :: Text }
\end{minted}
To associate a label with a review we can leverage the 
information present in the record type.
%
Specifically, we can specify that the only user allowed to modify such
a review is the owner of the review herself; and, we can specify that
the only users allowed to read such a review are the owner and other
reviewers of the same paper.
%
The latter declaration requires that we perform a lookup, using the
paper id of the current review, to find the other reviewers.
%
The code implementing this policy is given below.
%
\begin{minted}[frame=lines]{haskell}
policy :: HailsDB m => Review -> m DCLabel
policy rev = do
  let author = reviewOwner rev
  reviewers <- findReviewersOf $ reviewPaper rev
  makePolicy $ do
    readers ==> author \/ reviewers
    writers ==> author
\end{minted}
%
The function is self-explanatory; we only remark that the function
takes a \hs|Review| and returns a \hs|DCLabel| in a monad \hs|m| that
allows code to perform database actions (in this case the
\hs|findreviewersOf| action), a change from the original pure policies
of Hails.
%

We remark, that while, some care must be taken to ensure that the
specified policy is correct, the extend to understanding a security
policy in such LIO/Hails applications is limited to such functions.
%
It is these policy functions that the database system uses to label
reviews when a fetch, insert, or update is performed.
%
Indeed, the core of the conference review system does not manipulate
labels---high-level APIs make most of the DIFC details transparent.


